\begin{findingbox}{critical}
	%Weakly Encrypted Credentials in MGRemote Configuration File
	\textbf{Severity:} High\\
	\textbf{CVSS 3.1 Score:} \url{CVSS:3.0/AV:L/AC:L/PR:N/UI:N/S:U/C:H/I:H/A:N} 7.8 (High)\\
	\textbf{Affected Domain:} lock.htb\\[5pt]
	
	\textbf{Description and Root Cause:} \\
	During the assessment, an MGRemote configuration file was identified on the client system containing a password in its default encrypted form. The encryption mechanism used is weak and reversible, meaning that anyone with access to the file can easily recover the plaintext credentials. This issue stems from reliance on the product’s default configuration instead of implementing secure credential storage practices.
	\\
	
	\vspace{4pt}
	\textbf{Security Impact:} \\
	Credential exposure occurs when attackers gain access to a configuration file, enabling them to retrieve plaintext credentials. If these credentials belong to privileged accounts, it can lead to privilege escalation, giving the attacker elevated access within the system. Additionally, the compromised credentials may be reused across different systems, facilitating lateral movement and allowing attackers to navigate through the environment undetected. Even after remediation efforts, there is a persistence risk, as stolen credentials can be exploited if they are not rotated properly.
	\\
	
	\vspace{4pt}
	\textbf{Remediation:} 
	\begin{itemize}
		\item Remove or Replace Default Encryption: Avoid relying on the default reversible encryption provided by MGRemote. At a minimum, configure MGRemote with a strong, custom password rather than leaving credentials in the default encrypted form. Where possible, integrate secure storage mechanisms to prevent reversible encryption from being the sole protection.
		
		\item Use Strong Encryption/Secrets Management: Store credentials using a secure secrets manager or operating system–native secure storage (e.g., DPAPI, LSA secrets, or Vault-based solutions).
		
		\item Credential Rotation: Immediately rotate any accounts stored in the affected configuration files to prevent potential misuse.
		
		\item Least Privilege: Ensure the credentials used in MGRemote have minimal permissions required for functionality.
		
		\item Access Control: Restrict file system permissions to limit who can read or modify configuration files.
		
		\item Monitoring: Audit usage of the exposed accounts and enable alerting for suspicious or abnormal authentication attempts.
	\end{itemize}
	
	\vspace{4pt}
	\textbf{External References:}
	\begin{itemize}
		
		\item \url{https://cheatsheetseries.owasp.org/cheatsheets/Password_Storage_Cheat_Sheet.html}
		\item \url{https://attack.mitre.org/techniques/T1552/001/}
		
	\end{itemize}
	
\end{findingbox}

\subsection*{Credentials Files found.}
On the Document folder of the user ellen.freeman we found a configuration file.

\begin{figure}[H]
	\centering
	\includegraphics[width=0.7\linewidth]{assets/token}
	\caption[Decrypted Credentials]{Decrypted Credentials MGRemote}
	\label{fig:screenshot001}
\end{figure}
